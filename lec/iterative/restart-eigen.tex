\section{Restarting}

When discussing GMRES, we said that to keep storage under control,
we typically {\em restart} after a few steps.  The same technique
works for solving eigenvalue problems, but requires more care.
In particular, when solving a linear system, it made sense to restart
with a whole new Krylov subspace.  For eigenvalue problems,
{\em implicit} restarting is the norm.

The earliest versions of implicit restarting followed the strategy:
\begin{enumerate}
\item
  Build an initial Arnoldi decomposition
  \[
    A Q_m = Q_m H_m + \beta_m q_{m+1} e_m^T.
  \]
\item
  Do several steps of shifted QR iteration on the projected matrix
  $H_m$ to get a new decomposition
  \[
    A \tilde{Q}_m =
    \tilde{Q}_m \tilde{H}_m + \tilde{\beta_m} \tilde{q}_{m+1} e_m^T.
  \]
\item
  ``Cut back'' to a basis consisting of the first $p$ vectors
  of $\tilde{Q}_m$.
\end{enumerate}
The ``filter and cut back'' approach of the implicitly restarted
Arnoldi method involves some technical difficulties, but there is
good software available (the ARPACK code of Lehoucq and Sorensen).
This is the basis for the {\tt eigs} code in MATLAB.

A simpler method was introduced in 2002 by Pete Stewart,
the {\em Krylov-Schur} method.  The Krylov-Schur approach rests on
a more general decomposition than Arnoldi, a so-called {\em Krylov
decomposition}
\[
  A U_k = U_k B_k + u_{k+1} b_{k+1}^*;
\]
if the columns of $U_k$ are orthonormal, we call this an orthonormal
decomposition.  The idea of the Krylov-Schur method is to compute a
Schur decomposition of $B_k$, then sort the Schur decomposition to move
the unwanted Ritz values to the ``end,'' where they can be purged
by truncating the decomposition.  This approach avoids some of the
technical issues in previous implicit restarting methods that maintained
an Arnoldi decomposition throughout.
