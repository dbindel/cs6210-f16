\section{Krylov subspaces}

The {\em Krylov subspace} of dimension $k$ generated by
$A \in \bbR^{n \times n}$ and $b \in \bbR^n$ is
\[
  \mathcal{K}_k(A,b)
    = \operatorname{span}\{ b, Ab, \ldots, A^{k-1} b \}
    = \{ p(A) b : p \in \mathcal{P}_{k-1} \}.
\]
Krylov subspaces are a natural choice for subspace-based methods for
approximate linear solves, for two reasons:
\begin{itemize}
\item If all you are allowed to do with $A$ is compute matrix-vector
  products, and the only vector at hand is $b$, what else would you do?
\item The Krylov subspaces have excellent approximation properties.
\end{itemize}

Krylov subspaces have several properties that are worthy of comment.
Because the vectors $A^{j} b$ are proportional to the vectors obtained
in power iteration, one might reasonably (and correctly)
assume that the space quickly contains good approximations to the
eigenvectors associated with the largest magnitude eigenvalues.
Krylov subspaces are also {\em shift-invariant}, i.e. for any $\sigma$
\[
  \mathcal{K}_k(A-\sigma I, b) = \mathcal{K}_k(A,b).
\]
By choosing different shifts, we can see that the Krylov subspaces
tend to quickly contain not only good approximations to the eigenvector
associated with the largest magnitude eigenvalue, but to all
``extremal'' eigenvalues.

Most arguments about the approximation properties of Krylov subspaces
derive from the characterization of the space as all vectors $p(A) b$
where $p \in \mathcal{P}_{k-1}$ and from the spectral mapping theorem,
which says that if $A = V \Lambda V^{-1}$ then
$p(A) = V p(\Lambda) V^{-1}$.  Hence, the distance between
an arbitrary vector (say $d$) and the Krylov subspace is
\[
  \min_{p \in \mathcal{P}_{k-1}}
  \left\| V \left[ p(\Lambda) V^{-1} b - V^{-1} d \right] \right\|.
\]
As a specific example, suppose that we want to choose $\hat{x}$
in a Krylov subspace in order to minimize the residual $A \hat{x} - b$.
Writing $\hat{x} = p(A) b$, we have that we want to minimize
\[
  \|[A p(A)-I] b\| = \|q(A) b\|
\]
where $q(z)$ is a polynomial of degree at most $k$ such that $q(1) = 1$.
The best possible residual in this case is bounded by
\[
  \|q(A) b\| \leq \kappa(V) \|q(\Lambda)\| \|b\|,
\]
and so the relative residual can be bounded in terms of the condition
number of $V$ and the minimum value that can bound $q$ on the spectrum
of $A$ subject to the constraint that $q(0) = 1$.
