\section{GMRES}

The {\em generalized minimal residual} (GMRES) method of solving
linear systems works with general systems of linear equations.
Next to CG, it is probably the second-most popular of the Krylov
subspace iterations.

The GMRES method is so named because it chooses the solution from a
linear subspace that minimizes the (Euclidean) norm of the residual
over successive Krylov subspaces.  In terms of the Arnoldi
decompositions
\[
  AQ_k = Q_{k+1} \bar{H}_k,
\]
we have that $x_k = Q_k y_k$ where
\[
  y_k = \operatorname{argmin}_y \|\bar{H}_k y - \|b\| e_1\|^2.
\]
One can solve the Hessenberg least squares problem in $O(k^2)$
time, but this is generally a non-issue.  The true cost of GMRES
is in saving the basis (which can use memory very quickly) and in
keeping the basis orthogonal.

Unlike the CG method, alas, the GMRES method does not boil down to a
short recurrence through a sequence of clever tricks.  Consequently,
we generally cannot afford to run the iteration for many steps before
{\em restart}.  We usually denote the iteration with periodic
restarting every $m$ steps as GMRES$(m)$.  That is, at each step we
\begin{enumerate}
  \item Start with an initial guess $\hat{x}$ from previous steps.
  \item Form the residual $r = b-A\hat{x}$.
  \item Run $m$ steps of GMRES to approximately solve $Az = r$.
  \item Update $\hat{x} := \hat{x} + z$.
\end{enumerate}

The GMRES iteration is generally used with a preconditioner.
The common default is to use preconditioning on the left, i.e.
solve
\[
  M^{-1} A x = M^{-1} b;
\]
in this setting, GMRES minimizes not the original residual,
but the {\em preconditioned} residual.  To the extent that the
preconditioner reduces the condition number of the problem overall,
the norm of the preconditioned residual tends to be a better indicator
for forward error than the norm of the un-preconditioned residual.
Of course, one can also perform preconditioning on the right (i.e. changing
the unknown), or perform two-sided preconditioning.

The standard GMRES iteration (along with CG and almost every other
Krylov subspace iteration) assumes a single, fixed preconditioner.
But what if we want to try several preconditioners at once, or
perhaps to use Gauss-Southwell or a chaotic relaxation method for
preconditioning?  Or perhaps we want to use a variable number of
steps of some other iteration to precondition something like GMRES?
For this purpose, it is useful to consider the
{\em flexible GMRES} variant (FGMRES).  Though it no longer technically
is restricted to a Krylov subspace generated by a fixed matrix, the
FGMRES iteration looks very similar to the standard GMRES iteration;
we build an Arnoldi-like decomposition with the form
\[
  AZ_m = V_{m+1} \bar{H}_m
\]
and then compute updates as a linear combination of the columns of $Z_m$
by solving a least squares problem with $\bar{H}_m$.  But here, each
column of $Z_m$ looks like $z_j = M_j^{-1} v_j$ where each $M_j$ may
be different.

% CG vs GMRES
% Preconditioning
% Restarting
% Flexible GMRES
