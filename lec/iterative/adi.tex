\section{Alternating Direction Implicit}

The {\em alternating direction implicit} approach to the model problem
began life as an operator-splitting approach to solving a time-domain
diffusion problem.  At each step of an ordinary implicit time stepper
for the heat equation, one would solve a system of the form
\[
  (I+\Delta t T) x = b,
\]
where $\Delta t$ is small and $T$ is the 2D Laplacian operator.  But
note that if $T = T_x + T_y$, then
\[
  (I+\Delta t/2 T_x)(I+ \Delta t/2 T_y) = (I+\Delta t T + O(\Delta t)^2);
\]
hence, we commit only a small amount of error if instead of solving one
system with $T$ we solve two half-step systems involving $T_x$ and $T_y$,
respectively, where $T_x$ and $T_y$ are the discretizations of the
derivative operator in the $x$ and $y$ directions.  This is known as the
{\em alternating direction} method.  The implementation is relatively
straightforward:
\lstinputlisting{code/iter/sweep_adi.m}

In practice, cycling between several different versions of the shift parameter
(interpreted above as a time-step) can lead to very rapid convergence of
the ADI iteration.  This beautiful classical result, which has deep
connections to the Zolotarev problem from approximation theory, has taken
on renewed usefulness in modern control theory and model reduction,
where recent work has connected ADI-type methods for Sylvester equations
to various rational Krylov methods.

The ADI method and its relations have also garnered many citations over
the past 5--10 years because of their role as prior art for various
optimization methods, such as the ADMM method.
