\section{The need for model problems}

Direct methods for solving linear systems and eigenvalue problems are
(mostly) ``black box.''  We design algorithms that work well for a
broad category of problems with given structural properties; once we
understand the structure, there is often a reasonably routine choice
of solvers.  Of course, even for direct methods, it is not entirely
true that we get ``black box'' performance --- for example, the fill
in sparse direct factorization methods is highly dependent on the
sparsity structure of the matrix at hand.  Nonetheless, users of
sparse solvers can largely leave the details to specialists once they
understand the basic lay of the land.

For the remainder of the semester, we will focus on iterative solvers,
which are a different beast altogether.  Iterative solvers produce a
sequence of approximate solutions that (ideally) converge to the true
solution to a linear system or eigenvalue problem.  However, the rate
of convergence is highly dependent on both the iterative method and
the details of the problem.  Even when we are able to take advantage
of a good library of iterative solvers, there are often a wide variety
of methods to choose from and a large number of parameters that we need
to understand and tune to get good performance.

Because iterative methods are more problem-dependent than direct methods,
we will focus our presentation on a set of model problems that exhibit
characteristics common in many problems drawn from physical models.
We will also comment on other types of problem structures as we go along,
but will mostly leave the details to select homework problems.
