\section{Vectors}

A vector space (or linear space) is a set of vectors that can be added
or scaled in a sensible way -- that is, addition is associative and
commutative and scaling is distributive.  We will generally denote
vector spaces by script letters (e.g.~$\mathcal{V}, \mathcal{W}$),
vectors by lower case Roman letters (e.g.~$v, w$), and scalars by
lower case Greek letters (e.g.~$\alpha, \beta$).  But we feel free to
violate these conventions according to the dictates of our conscience
or in deference to other conflicting conventions.

There are many types of vector spaces.  Apart from the ubiquitous
spaces $\bbR^n$ and $\bbC^n$, the most common vector spaces in applied
mathematics are different types of function spaces.  These include
\begin{align*}
  \mathcal{P}_d &=
  \{ \mbox{polynomials of degree at most $d$} \}; \\
  \mathcal{V}^* &=
  \{ \mbox{linear functions $\mathcal{V} \rightarrow \bbR$ (or $\bbC$)} \}; \\
  L(\mathcal{V}, \mathcal{W}) &=
  \{ \mbox{linear maps $\calV \rightarrow \calW$} \}; \\
  \mathcal{C}^k(\Omega) &=
  \{\mbox{ $k$-times differentiable functions on a set $\Omega$} \};
\end{align*}
and many more.  We compute with vectors in $\bbR^n$ and $\bbC^n$,
which we represent concretely by tuples of numbers in memory, usually
stored in sequence.  To keep a broader perspective, though, we will
also frequently describe examples involving the polynomial spaces
$\mathcal{P}_d$.

\subsection{Spanning sets and bases}

We often think of a matrix as a set of vectors
\[
  V = \begin{bmatrix} v_1 & v_2 & \ldots & v_n \end{bmatrix}.
\]
The {\em range space} $\mathcal{R}(V)$ or the {\em span}
$\operatorname{sp}\{ v_j \}_{j=1}^n$ is the set of vectors
$\{ V c : c \in \bbR^n \}$ (or $\bbC^n$
for complex spaces).  The vectors are {\em linearly independent}
if any vector in the span has a unique representation as a linear
combination of the spanning vectors; equivalently, the vectors are
linearly independent if there is no linear combination of the
spanning vectors that equals zero except the one in which all
coefficients are zero.  A linearly independent set of vectors is
a {\em basis} for a space $\mathcal{V}$ if
$\operatorname{sp}\{v_j\}_{j=1}^n = \mathcal{V}$; the number of
basis vectors $n$ is the {\em dimension} of the space.  Spaces
like $\mathcal{C}^k(\Omega)$ do not generally have a finite basis;
they are {\em infinite-dimensional}.  We will focus on
finite-dimensional spaces in the class, but it is useful to know
that there are interesting infinite-dimensional spaces in the
broader world.

The {\em standard basis in $\bbR^n$} is the set of column vectors
$e_j$ with a one in the $j$th position and zeros elsewhere:
\[
  I = \begin{bmatrix} e_1 & e_2 & \ldots & e_n \end{bmatrix}.
\]
Of course, this is not the only basis.  Any other set of $n$ linearly
independent vectors in $\bbR^n$ is also a basis
\[
  V = \begin{bmatrix} v_1 & v_2 & \ldots & v_n \end{bmatrix}.
\]
The matrix $V$ formed in this way is invertible, with multiplication
by $V$ corresponding to a change from the $\{v_j\}$ basis into the
standard basis and multiplication by $V^{-1}$ corresponding to a map
from the $\{v_j\}$ basis to the standard basis.

We will use matrix-like notation to describe bases and spanning sets
even when we deal with spaces other than $\bbR^n$.  For example, a
common basis for $\mathcal{P}_d$ is the {\em power basis}
\[
  X = \begin{bmatrix} 1 & x & x^2 & \ldots & x^d \end{bmatrix}.
\]
Each ``column'' in $X$ is really a function of one variable ($x$), and
matrix-vector multiplication with $X$ represents a map from a
coefficient vector in $\bbR^{d+1}$ to a polynomial in $\mathcal{P}_d$.
That is, we write polynomials $p \in \mathcal{P}_d$ in terms of the
basis as
\[
  p = Xc = \sum_{j=0}^d c_j x^j
\]
and, we think of computing the coefficients from the
abstract polynomial via a formal inverse:
\[
  c = X^{-1} p.
\]
We typically think of a map like $Y^* = X^{-1}$ in terms of ``rows''
\[
  Y^* = \begin{bmatrix} y_0^* \\ y_1^* \\ \vdots \\ y_d^* \end{bmatrix}
\]
where each row $y_j^*$ is a {\em linear functional} or {\em dual
  vector} (i.e.~linear mappings from the vector space to the real or
complex numbers).  Collectively, $\{y_0^*, \ldots, y_d^*\}$ are the
{\em dual basis} to $\{1, x, \ldots, x^d\}$.

The power basis is not the only basis for $\mathcal{P}_d$.  Other
common choices include Newton or Lagrange polynomials with respect to
a set of points, which you may have seen in another class such as CS
4210.  In this class, we will sometimes use the Chebyshev\footnote{
  Pafnuty Chebyshev %(Чебышёв)
  was a nineteenth century Russian
  mathematician, and his name has been transliterated from the
  Cyrillic alphabet into the Latin alphabet in several different ways.
  We inherit our usual spelling from one of the French
  transliterations, but the symbol $T$ for the polynomials comes from
  the German transliteration Tschebyscheff.  } polynomial basis $\{
T_j(x) \}$ given by the recurrence
\begin{align*}
  T_0(x) &= 1 \\
  T_1(x) &= x \\
  T_{j+1}(x) &= 2 x T_{j}(x) - T_{j-1}(x), \quad j \geq 1,
\end{align*}
and the Legendre polynomial basis
$\{ P_j(x) \}$, given by the recurrence
\begin{align*}
  P_0(x) &= 1 \\
  P_1(x) &= x \\
  (j+1) P_{j+1}(x) &= (2j+1) x P_j(x) - j P_{j-1}(x).
\end{align*}
As we will see over the course of the semester, sometimes the
``obvious'' choice of basis (e.g.~the standard basis in $\bbR^n$ or
the power basis in $\mathcal{P}_d$) is not the best choice for
numerical computations.

\subsection{Vector norms}

A {\em norm} $\|\cdot\|$ measures vector lengths.  It is
positive definite, homogeneous, and sub-additive:
\begin{align*}
  \|v\| & \geq 0 \mbox{ and } \|v\| = 0 \mbox{ iff } v = 0 \\
  \|\alpha v\| &= |\alpha| \|v\| \\
  \|u+v\| & \leq \|u\| + \|v\|.
\end{align*}
The three most common vector norms we work with in $\bbR^n$ are the
Euclidean norm (aka the 2-norm), the $\infty$-norm (or max norm),
and the $1$-norm:
\begin{align*}
  \|v\|_2 &= \sqrt{\sum_j |v_j|^2} \\
  \|v\|_\infty &= \max_j |v_j| \\
  \|v\|_1 &= \sum_j |v_j|
\end{align*}

Many other norms can be related to one of these three norms.  In
particular, a ``natural'' norm in an abstract vector space will often
look strange in the corresponding concrete representation with respect
to some basis function.  For example, consider the vector space of
polynomials with degree at most 2 on $[-1,1]$.  This space also has
a natural Euclidean norm, max norm, and $1$-norm; for a given
polynomial $p(x)$ these are
\begin{align*}
  \|p\|_2 &= \sqrt{ \int_{-1}^1 |p(x)|^2 \, dx } \\
  \|p\|_\infty &= \max_{x \in [-1,1]} |p(x)| \\
  \|p\|_1 &= \int_{-1}^1 |p(x)| \, dx.
\end{align*}
But when we write $p(x)$ in terms of the coefficient vector with
respect to the power basis (for example), the max norm of the
polynomial is not the same as the max norm of the coefficient vector.
In fact, if we consider a polynomial $p(x) = c_0 + c_1 x$, then
the max norm of the polynomial $p$ is the same as the one-norm of the
coefficient vector --- the proof of which is left as a useful exercise
to the reader.

In a finite-dimensional vector space, all norms are {\em equivalent}:
that is, if $\|\cdot\|$ and $\vertiii{\cdot}$ are two norms on the
same finite-dimensional vector space, then there exist constants $c$
and $C$ such that for any $v$ in the space,
\[
  c \|v\| \leq \vertiii{v} \leq C \|v\|.
\]
Of course, there is no guarantee that the constants are small!

An {\em isometry} is a mapping that preserves vector norms.
For $\bbR^n$, the only isometries for the 1-norm and the $\infty$-norm
are permutations.  For Euclidean space, though, there is a much
richer set of isometries, represented by the orthogonal matrices
(matrices s.t.~$Q^* Q = I$).

% Vector spaces
% - Abstract picture
% - R^n
% - Polynomials
% - Norms

\subsection{Inner products}

An {\em inner product} $\langle \cdot, \cdot \rangle$
is a function from two vectors into the real
numbers (or complex numbers for an complex vector space).  It is
positive definite, linear in the first slot, and symmetric (or
Hermitian in the case of complex vectors); that is:
\begin{align*}
  \langle v, v \rangle & \geq 0 \mbox{ and }
  \langle v, v \rangle = 0 \mbox{ iff } v = 0 \\
%
  \langle \alpha u, w \rangle &= \alpha \langle u, w \rangle
  \mbox{ and }
  \langle u+v, w \rangle = \langle u, w \rangle + \langle v, w \rangle \\
%
  \langle u, v \rangle &= \overline{\langle v, u \rangle},
\end{align*}
where the overbar in the latter case corresponds to complex
conjugation.  A vector space with an inner product is sometimes
called an {\em inner product space} or a {\em Euclidean space}.

Every inner product defines a corresponding norm
\[
  \|v\| = \sqrt{ \langle v, v \rangle}
\]
The inner product and the associated norm satisfy
the {\em Cauchy-Schwarz} inequality
\[
  \langle u, v \rangle \leq \|u\| \|v\|.
\]
The {\em standard inner product} on $\bbC^n$ is
\[
  x \cdot y = y^* x = \sum_{j=1}^n \bar{y}_j x_j.
\]
But the standard inner product is not the only inner product,
just as the standard Euclidean norm is not the only norm.

Just as norms allow us to reason about size, inner products let us
reason about angles.  In particular, we define the cosine of
the angle $\theta$ between nonzero vectors $v$ and $w$ as
\[
  \cos \theta = \frac{\langle v, w \rangle}{\|v\| \|w\|}.
\]

Returning to our example of a vector space of polynomials,
the standard $L^2([-1,1])$ inner product is
\[
  \langle p, q \rangle_{L^2([-1,1])} = \int_{-1}^1 p(x) \bar{q}(x) \, dx.
\]
If we express $p$ and $q$ with respect to a basis (e.g.~the power
basis), we fine that we can represent this inner product via a
symmetric positive definite matrix.  For example,
let $p(x) = c_0 + c_1 x + c_2 x^2$ and let
$q(x) = d_0 + d_1 x + d_2 x^2$.  Then
\[
  \langle p, q \rangle_{L^2([-1,1])} =
  \begin{bmatrix} d_0 \\ d_1 \\ d_2 \end{bmatrix}^*
  \begin{bmatrix}
    a_{00} & a_{01} & a_{02} \\
    a_{10} & a_{11} & a_{12} \\
    a_{20} & a_{21} & a_{22}
  \end{bmatrix}
  \begin{bmatrix} c_0 \\ c_1 \\ c_2 \end{bmatrix} =
  d^* A c = \langle c, d \rangle_A
\]
where
\[
a_{ij} = \int_{-1}^1 x^{i-1} x^{j-1} \, dx =
\begin{cases}
  2/(i+j-1), & i+j \mbox{ even} \\
  0, & \mbox{ otherwise}
\end{cases}
\]
The symmetric positive definite matrix $A$ is what is sometimes called
the {\em Gram matrix} for the basis $\{1, x, x^2\}$.

We say two vectors $u$ and $v$ are {\em orthogonal} with respect to an
inner product if $\langle u, v \rangle = 0$.  If $u$ and $v$ are
orthogonal, we have the {\em Pythagorean theorem}:
\begin{align*}
  \|u+v\|^2
  = \langle u + v, u + v \rangle
  = \langle u, u \rangle + \langle v, u \rangle + \langle u, v \rangle
  + \langle v, v \rangle
  = \|u\|^2 + \|v\|^2
\end{align*}
Two vectors $u$ and $v$ are {\em orthonormal} if they are orthogonal
with respect to the inner product and have unit length in the associated
norm.  When we work in an inner product space, we often use an
{\em orthonormal basis}, i.e.~a basis in which all the vectors are
orthonormal.  For example, the normalized Legendre polynomials
\[
  \sqrt{\frac{2}{2j+1}} P_j(x)
\]
form orthonormal bases for the $\mathcal{P}_d$ inner product with
respect to the $L^2([-1,1])$ inner product.
