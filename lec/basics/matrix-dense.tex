\section{Dense matrix basics}

% Level 1-2 BLAS, locality, blocking ideas, memory layout
% Asides regarding JIT compilation in MATLAB

There is one common data structure for dense vectors: we store
the vector as a sequential array of memory cells.  In contrast,
there are {\em two} common data structures for general dense matrices.
In MATLAB (and Fortran), matrices are stored in {\em column-major} form.
For example, an array of the first four positive integers interpreted
as a two-by-two column major matrix represents the matrix
\[
    \begin{bmatrix} 1 & 3 \\ 2 & 4 \end{bmatrix}.
\]
The same array, when interpreted as a {\em row-major} matrix, represents
\[
    \begin{bmatrix} 1 & 2 \\ 3 & 4 \end{bmatrix}.
\]
Unless otherwise stated, we will assume all dense matrices are represented
in column-major form for this class.  As we will see, this has some
concrete effects on the efficiency of different types of algorithms.

\subsection{The BLAS}

The {\em Basic Linear Algebra Subroutines} (BLAS) are a standard library
interface for manipulating dense vectors and matrices.  There are three
{\em levels} of BLAS routines:
\begin{itemize}
\item {\bf Level 1}:
  These routines act on vectors, and include operations
  such scaling and dot products.  For vectors of length $n$,
  they take $O(n^1)$ time.
\item {\bf Level 2:}
  These routines act on a matrix and a vector, and include operations
  such as matrix-vector multiplication and solution of triangular systems
  of equations by back-substitution.  For $n \times n$ matrices and length
  $n$ vectors, they take $O(n^2)$ time.
\item {\bf Level 3:}
  These routines act on pairs of matrices, and include operations such
  as matrix-matrix multiplication.  For $n \times n$ matrices, they
  take $O(n^3)$ time.
\end{itemize}
All of the BLAS routines are superficially equivalent to algorithms
that can be written with a few lines of code involving one, two, or
three nested loops (depending on the level of the routine).  Indeed,
except for some refinements involving error checking and scaling for
numerical stability, the reference BLAS implementations involve
nothing more than these basic loop nests.  But this simplicity is
deceptive --- a surprising amount of work goes into producing high
performance implementations.

\subsection{Locality and memory}

When we analyze algorithms, we often reason about their complexity
abstractly, in terms of the scaling of the number of operations required
as a function of problem size.  In numerical algorithms, we typically
measure {\em flops} (short for floating point operations).  For example,
consider the loop to compute the dot product of two vectors:
\lstinputlisting{code/mydot.m}
Because it takes $n$ additions and $n$ multiplications, we say this code
takes $2n$ flops, or (a little more crudely) $O(n)$ flops.

On modern machines, though, counting flops is at best a crude way
to reason about how run times scale with problem size.  This is because
in many computations, the time to do arithmetic is dominated by the time
to fetch the data into the processor!  A detailed discussion of modern
memory architectures is beyond the scope of these notes, but there are
at least two basic facts that everyone working with matrix computations
should know:
\begin{itemize}
\item
  Memories are optimized for access patterns with {\em spatial locality}:
  it is faster to access entries of memory that are close to each
  other (ideally in sequential order) than to access memory entries that
  are far apart.  Beyond the memory system, sequential access patterns
  are good for {\em vectorization}, i.e.~for scheduling work to be done
  in parallel on the vector arithmetic units
  that are present on essentially all modern processors.
\item
  Memories are optimized for access patterns with {\em temporal locality};
  that is, it is much faster to access a small amount of data repeatedly
  than to access large amounts of data.
\end{itemize}

The main mechanism for optimizing access patterns with temporal locality
is a system of {\em caches}, fast and (relatively) small memories that can
be accessed more quickly (i.e.~with lower latency) than the main memory.
To effectively use the cache, it is helpful if the {\em working set}
(memory that is repeatedly accessed) is smaller than the cache size.
For level 1 and 2 BLAS routines, the amount of work is proportional to
the amount of memory used, and so it is difficult to take advantage of
the cache.  On the other hand, level 3 BLAS routines do $O(n^3)$ work
with $O(n^2)$ data, and so it is possible for a clever level 3 BLAS
implementation to effectively use the cache.

\subsection{Matrix-vector multiply}

Let us start with a very simple \matlab\ program for matrix-vector
multiplication:
\lstinputlisting{code/matvec1.m}
We could just as well have switched the order of the $i$ and $j$ loops
to give us a column-oriented rather than row-oriented version of the algorithm.
Let's consider these two variants, written more compactly:
\lstinputlisting{code/matvec2_row.m}
\lstinputlisting{code/matvec2_col.m}

It's not too surprising that the builtin matrix-vector multiply routine in
\matlab\ runs faster than either of our {\tt matvec2} variants, but there
are some other surprises lurking.  Try timing each of these matrix-vector
multiply methods for random square matrices of size 4095, 4096, and 4097,
and see what happens.  Note that you will want to run each code many times
so that you don't get lots of measurement noise from finite timer granularity;
for example, try
\begin{lstlisting}
tic;          % Start timer
for i = 1:100 % Do enough trials that it takes some time
  % ...         Run experiment here
end
toc           % Stop timer
\end{lstlisting}

\begin{figure} \label{fig:matvec-time}
  \begin{tikzpicture}
\begin{axis}[width=\textwidth,xlabel={$n$},ylabel={GFlop/s},xmin=0,ymin=0]

  \addplot table [x={n}, y={matvec0}] {data/matvec_time.dat};
  \addlegendentry{Default};

  \addplot table [x={n}, y={matvec1}] {data/matvec_time.dat};
  \addlegendentry{Two loops};

  \addplot table [x={n}, y={matvec2_row}] {data/matvec_time.dat};
  \addlegendentry{Row-oriented};

  \addplot table [x={n}, y={matvec2_col}] {data/matvec_time.dat};
  \addlegendentry{Col-oriented};
\end{axis}
\end{tikzpicture}

  \caption{Timing of four matrix-vector multiply implementations.
    In each case, we report the effective time in GFLop/s.
    The line labeled ``default'' is the built-in MATLAB matvec.}
\end{figure}

On my machine (a late MacBook Pro 13 inch, late 2013 model) and
using \matlab\ 2016a, we show the GFlop rates (billions of flops/second)
for the three matrix multiply routines in Figure~\ref{fig:matvec-time}.
There are a few things to notice:
\begin{itemize}
\item The performance of the built-in multiply far exceeds that
  of any of the manual implementations.
\item The peak performance occurs for moderate size matrices where
  the matrix fits into cache, but there is enough work to hide
  the \matlab\ loop overheads.
\item The time required for the built-in routine varies dramatically
  (due to so-called {\em conflict misses}) when the dimension is
  a multiple of a large integer power of two.
\item For $n = 1024$, the column-oriented version (which has good spatial
  locality) is $10\times$ faster than the row-oriented code,
  and $45\times$ faster than the two nested loop version.
\end{itemize}
If you are so inclined, consider yourself encouraged to repeat the
experiment using your favorite compiled language to see if any of the
general trends change significantly.

% Matrix multiplication and blocking
% Memory access and vectorization issues; BLAS routines
% Matrix representation

\subsection{Matrix-matrix multiply}

The classic algorithm to compute $C := C + AB$ involves
three nested loops
\lstinputlisting{code/matmul_3loop.m}
This is sometimes called an {\em inner product} variant of
the algorithm, because the innermost loop is computing a dot
product between a row of $A$ and a column of $B$.  But
addition is commutative and associative, so we can sum the
terms in a matrix-matrix product in any order and get the same
result.  And we can interpret the orders!  A non-exhaustive
list is:
\begin{itemize}
\item {\tt ij(k)} or {\tt ji(k)}: Compute entry $c_{ij}$ as a
  product of row $i$ from $A$ and column $j$ from $B$
  (the {\em inner product} formulation)
\item {\tt k(ij)}: $C$ is a sum of outer products of column $k$
  of $A$ and row $k$ of $B$ for $k$ from $1$ to $n$
  (the {\em outer product} formulation)
\item {\tt i(jk)} or {\tt i(kj)}: Each row of $C$ is a row of
  $A$ multiplied by $B$
\item {\tt j(ij)} or {\tt j(ki)}: Each column of $C$ is $A$
  multiplied by a column of $C$
\end{itemize}
At this point, we could write down all possible loop orderings
and run a timing experiment, similar to what we did with
matrix-vector multiplication.  But the truth is that high-performance
matrix-matrix multiplication routines use another access pattern
altogether, involving more than three nested loops, and we will
describe this now.

\subsection{Blocking and performance}

The basic matrix multiply outlined in the previous section will
usually be at least an order of magnitude slower than a well-tuned
matrix multiplication routine.  There are several reasons for this
lack of performance, but one of the most important is that the basic
algorithm makes poor use of the {\em cache}.
Modern chips can perform floating point arithmetic operations much
more quickly than they can fetch data from memory; and the way that
the basic algorithm is organized, we spend most of our time reading
from memory rather than actually doing useful computations.
Caches are organized to take advantage of {\em spatial locality},
or use of adjacent memory locations in a short period of program execution;
and {\em temporal locality}, or re-use of the same memory location in a
short period of program execution.  The basic matrix multiply organizations
don't do well with either of these.
A better organization would let us move some data into the cache
and then do a lot of arithmetic with that data.  The key idea behind
this better organization is {\em blocking}.

When we looked at the inner product and outer product organizations
in the previous sections, we really were thinking about partitioning
$A$ and $B$ into rows and columns, respectively.  For the inner product
algorithm, we wrote $A$ in terms of rows and $B$ in terms of columns
\[
  \begin{bmatrix} a_{1,:} \\ a_{2,:} \\ \vdots \\ a_{m,:} \end{bmatrix}
  \begin{bmatrix} b_{:,1} & b_{:,2} & \cdots & b_{:,n} \end{bmatrix},
\]
and for the outer product algorithm, we wrote $A$ in terms of colums
and $B$ in terms of rows
\[
  \begin{bmatrix} a_{:,1} & a_{:,2} & \cdots & a_{:,p} \end{bmatrix}
  \begin{bmatrix} b_{1,:} \\ b_{2,:} \\ \vdots \\ b_{p,:} \end{bmatrix}.
\]
More generally, though, we can think of writing $A$ and $B$ as
{\em block matrices}:
\begin{align*}
  A &=
  \begin{bmatrix}
    A_{11} & A_{12} & \ldots & A_{1,p_b} \\
    A_{21} & A_{22} & \ldots & A_{2,p_b} \\
    \vdots & \vdots &       & \vdots \\
    A_{m_b,1} & A_{m_b,2} & \ldots & A_{m_b,p_b}
  \end{bmatrix} \\
  B &=
  \begin{bmatrix}
    B_{11} & B_{12} & \ldots & B_{1,p_b} \\
    B_{21} & B_{22} & \ldots & B_{2,p_b} \\
    \vdots & \vdots &       & \vdots \\
    B_{p_b,1} & B_{p_b,2} & \ldots & B_{p_b,n_b}
  \end{bmatrix}
\end{align*}
where the matrices $A_{ij}$ and $B_{jk}$ are compatible for matrix
multiplication.  Then we we can write the submatrices of $C$ in terms
of the submatrices of $A$ and $B$
\[
  C_{ij} = \sum_k A_{ij} B_{jk}.
\]

\subsection{The lazy man's approach to performance}

An algorithm like matrix multiplication seems simple, but there is a
lot under the hood of a tuned implementation, much of which has to do
with the organization of memory.  We often get the best ``bang for our
buck'' by taking the time to formulate our algorithms in block terms,
so that we can spend most of our computation inside someone else's
well-tuned matrix multiply routine (or something similar).  There are
several implementations of the Basic Linear Algebra Subroutines
(BLAS), including some implementations provided by hardware vendors
and some automatically generated by tools like ATLAS.  The best BLAS
library varies from platform to platform, but by using a good BLAS
library and writing routines that spend a lot of time in {\em level 3}
BLAS operations (operations that perform $O(n^3)$ computation on
$O(n^2)$ data and can thus potentially get good cache re-use), we can
hope to build linear algebra codes that get good performance across
many platforms.

This is also a good reason to use \matlab: it uses pretty good BLAS libraries,
and so you can often get surprisingly good performance from it for the types
of linear algebraic computations we will pursue.
