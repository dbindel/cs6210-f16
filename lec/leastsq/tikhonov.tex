\section{Tikhonov}

Another approach is to say that we want a model in which the
coefficients are not too large.  To accomplish this, we add
a penalty term to the usual least squares problem:
\[
  \mbox{minimize } \|Ax-b\|^2 + \lambda^2 \|x\|^2.
\]
Equivalently, we can write
\[
\mbox{minimize } \left\|
\begin{bmatrix} A \\ \lambda I \end{bmatrix} x -
\begin{bmatrix} b \\ 0 \end{bmatrix}
\right\|^2,
\]
which leads to the regularized version of the normal equations
\[
  (A^T A + \lambda^2 I) x = A^T b.
\]
In some cases, we may want to regularize with a more general
norm $\|x\|_M^2 = x^T M x$ where $M$ is symmetric and positive
definite, which leads to the regularized equations
\[
  (A^T A + \lambda^2 M) x = A^T b.
\]
If we want to incorporate prior information that pushes $x$
toward some initial guess $x_0$, we may pose the least squares
problem in terms of $z = x-x_0$ and use some form of Tikhonov
regularization.  If we know of no particular problem structure
in advance, the standard choice of $M = I$ is a good default.

It is useful to compare the usual least squares solution to the
regularized solution via the SVD.  If $A = U \Sigma V^T$ is the
economy SVD, then
\begin{align*}
  x_{LS} &= V \Sigma^{-1} U^T b \\
  x_{Tik} &= V f(\Sigma)^{-1} U^T b
\end{align*}
where
\[
  f(\sigma) = \frac{1}{\sqrt{\sigma^{-1} + \lambda^2}}.
\]
This {\em filter} of the inverse singular values affects the larger
singular values only slightly, but damps the effect of very small
singular values.
