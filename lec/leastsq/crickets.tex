\section{Cricket chirps: an example}

Did you know that you can estimate the temperature by listening to the
rate of chirps?  The data set in Table~\ref{table1}%
\footnote{Data set originally attributed to
  \url{http://mste.illinois.edu}}.  represents measurements of the
number of chirps (over 15 seconds) of a striped ground cricket at
different temperatures measured in degrees Farenheit.  A plot
(Figure~\ref{fig2}) shows that the two are roughly correlated: the
higher the temperature, the faster the crickets chirp.  We can
quantify this by attempting to fit a linear model
\[
  \mbox{temperature} = \alpha \cdot \mbox{chirps} + \mbox{beta} + \epsilon
\]
where $\epsilon$ is an error term.  To solve this problem by linear
regression, we minimize the Euclidean norm of the residual
\[
  r = b-Ax
\]
where
\begin{align*}
  b_{i} &= \mbox{temperature in experiment } i \\
  A_{i1} &= \mbox{chirps in experiment } i \\
  A_{i2} &= 1 \\
  x &= \begin{bmatrix} \alpha \\ \beta \end{bmatrix}
\end{align*}
MATLAB and Octave are capable of solving least squares problems using
the backslash operator; that is, if {\tt chirps} and {\tt temp} are
column vectors in MATLAB, we can solve this regression problem as
\begin{lstlisting}
  A = [chirps, ones(ndata,1)];
  x = A\temp;
\end{lstlisting}
The algorithms underlying that backslash operation will make up
most of the next lecture.

\begin{figure}
  \begin{center}
    \begin{tikzpicture}
      \begin{axis}[xlabel={Chirps},ylabel={Degrees},grid=major]
        \addplot[only marks] table {leastsq/cricket.dat};
        \addplot table[x=chirp,y=fit]{leastsq/cricket.dat};
      \end{axis}
    \end{tikzpicture}
  \end{center}
  \caption{Cricket chirps vs.~temperature and a model fit via
    linear regression.}
  \label{fig2}
\end{figure}

\begin{table}
  \small
  \begin{tabular}{l|cccccccccccccccccc}
    Chirp &
    20& 16& 20& 18& 17& 16& 15& 17& 15& 16& 15& 17& 16& 17& 14 \\
    Temp &
    89& 72& 93& 84& 81& 75& 70& 82& 69& 83& 80& 83& 81& 84& 76
  \end{tabular}
  \caption{Cricket data: Chirp count over a 15 second period vs.~temperature
    in degrees Farenheit.}
  \label{table1}
\end{table}

In more complex examples, we want to fit a model involving more than
two variables.  This still leads to a linear least squares problem,
but one in which $A$ may have more than one or two columns.  As we
will see later in the semester, we also use linear least squares
problems as a building block for more complex fitting procedures,
including fitting nonlinear models and models with more complicated
objective functions.
