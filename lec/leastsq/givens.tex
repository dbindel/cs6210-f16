\section{Givens rotations}

Householder reflections are one of the standard orthogonal
transformations used in numerical linear algebra.  The other standard
orthogonal transformation is a {\em Givens rotation}:
\[
  G = \begin{bmatrix}
    c & -s \\
    s & c
  \end{bmatrix}.
\]
where $c^2 + s^2 = 1$.  Note that
\[
  G = \begin{bmatrix}
    c & -s \\
    s & c
  \end{bmatrix}
  \begin{bmatrix}
    x \\ y
  \end{bmatrix} =
  \begin{bmatrix}
    cx - sy \\
    sx + cy
  \end{bmatrix}
\]
so if we choose
\begin{align*}
  s &= \frac{-y}{\sqrt{x^2 + y^2}}, &
  c &= \frac{x}{\sqrt{x^2+y^2}}
\end{align*}
then the Givens rotation introduces a zero in the second column.
More generally, we can transform a vector in $\bbR^m$ into a vector
parallel to $e_1$ by a sequence of $m-1$ Givens rotations, where
the first rotation moves the last element to zero, the second rotation
moves the second-to-last element to zero, and so forth.

For some applications, introducing zeros one by one is very
attractive.  In some places, you may see this phrased as a contrast
between algorithms based on Householder reflections and those based on
Givens rotations, but this is not quite right.  Small Householder
reflections can be used to introduce one zero at a time, too.
Still, in the general usage, Givens rotations seem to be the more
popular choice for this sort of local introduction of zeros.
