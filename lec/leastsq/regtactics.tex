\section{Tradeoffs and tactics}

All four of the regularization approaches we have described are used
in practice, and each has something to recommend it.  The pivoted QR
approach is relatively inexpensive, and it results in a model that
depends on only a few factors.  If taking the measurements to compute
a prediction costs money --- or even costs storage or bandwidth for
the factor data! --- such a model may be to our advantage.  The
Tikhonov approach is likewise inexpensive, and has a nice Bayesian
interpretation (though we didn't talk about it).  The truncated SVD
approach involves the best approximation rank $k$ approximation to the
original factor matrix, and can be interpreted as finding the $k$ best
factors that are linear combinations of the original measurements.
The $\ell_1$ approach again produces models with sparse coefficients;
but unlike QR with column pivoting, the $\ell_1$ regularized solutions
incorporate information about the vector $b$ along with the matrix $A$.

So which regularization approach should one use?  In terms of
prediction quality, all can provide a reasonable deterrent against
ill-posedness and overfitting due to highly correlated factors.  Also,
all of the methods described have a parameter (the number of retained
factors, or a penalty parameter $\lambda$) that governs the tradeoff
between how well-conditioned the fitting problem will be and the
increase in bias that naturally comes from looking at a smaller class
of models.  Choosing this tradeoff intelligently may be rather more
important than the specific choice of regularization strategy.  A
detailed discussion of how to make this tradeoff is beyond the scope
of the class; but we will see some of the computational tricks
involved in implementing specific strategies for choosing
regularization parameters before we are done.
