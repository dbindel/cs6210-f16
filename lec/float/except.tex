\section{Exceptions}

We say there is an {\em exception} when the floating point result is
not an ordinary value that represents the exact result.  The most
common exception is {\em inexact} (i.e. some rounding was needed).
Other exceptions occur when we fail to produce a normalized floating
point number.  These exceptions are:
\begin{description}
\item[Underflow:]
  An expression is too small to be represented as a normalized floating
  point value.  The default behavior is to return a subnormal.
\item[Overflow:]
  An expression is too large to be represented as a floating point
  number.  The default behavior is to return {\tt inf}.
\item[Invalid:]
  An expression evaluates to Not-a-Number (such as $0/0$)
\item[Divide by zero:]
  An expression evaluates ``exactly'' to an infinite value
  (such as $1/0$ or $\log(0)$).
\end{description}
When exceptions other than inexact occur, the usual ``$1 + \delta$''
model used for most rounding error analysis is not valid.

An important feature of the floating point standard is that an
exception should {\em not} stop the computation by default.  This
is part of why we have representations for infinities and NaNs:
the floating point system is {\em closed} in the sense that every
floating point operation will return some result in the floating
point system.  Instead, by default, an exception is {\em flagged}
as having occurred\footnote{%
There is literally a register inside the computer with a set of
flags to denote whether an exception has occurred in a given
chunk of code.  This register is highly problematic, as it
represents a single, centralized piece of global state.  The
treatment of the exception flags --- and of exceptions generally ---
played a significant role in the debates leading up to the last revision
of the IEEE 754 floating point standard,
and I would be surprised if they are not playing a role again in the
current revision of the standard.
}.  An actual exception (in the sense of hardware
or programming language exceptions) occurs only if requested.
