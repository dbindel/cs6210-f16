\section{Sylvester equations}

The {\em Sylvester equation} (or the special case of
the {\em Lyapunov equation}) is a matrix equation of the form
\[
  AX + XB = C
\]
where $A \in \bbR^{m \times m}, B \in \bbR^{n \times n}, B \in \bbR^{m \times n}$, are known,
and $X \in \bbR^{m \times n}$ is to be determined.  The Sylvester
equation has important applications in control theory, and also plays
a prominent role in the theory of several classes of structured matrices.

On the surface of it,
this is a simple system: the expressions $AX$ and $XB$ are just linear
in the elements of $X$, after all.  Indeed, we can rewrite the system
as
\[
  (I \otimes A + B^T \otimes I)
  \operatorname{vec}(X) =
  \operatorname{vec}(C),
\]
where
\[
  \operatorname{\vec}(\begin{bmatrix} x_1 & \ldots & x_n \end{bmatrix}) =
  \begin{bmatrix} x_1 \\ \vdots \\ x_n \end{bmatrix}
\]
is a vector of length $mn$ composed by listing the elements of $X$ in
column-major order, and the {\em Kronecker product} is defined by
\[
  F \otimes G =
  \begin{bmatrix}
    f_{11} G & f_{12} G & \ldots \\
    f_{21} G & f_{22} G & \ldots \\
    \vdots & \vdots & \ddots
  \end{bmatrix}.
\]
Alas, solving this matrix equation by Gaussian elimination would cost
$O((mn)^3)$.  Can we do better?

The {\em Bartels-Stewart} algorithm is a clever approach to the problem
that takes only $O(\max(m,n)^3)$ time.  The key is to compute the
Schur factorizations
\begin{align*}
  A &= U_A T_A U_A^* &
  B &= U_B T_B U_B^*
\end{align*}
from which we obtain
\[
  T_A \tilde{X} + \tilde{X} T_B = \tilde{C}
\]
where $\tilde{X} = U_A^* X U_B$ and $\tilde{C} = U_A^* C U_B$.
Column $j$ of this system of equations can be written as
\[
  (T_A + t_{B,kk} I) \tilde{x}_k =
  \tilde{c}_k - \sum_{j=1}^{k-1} \tilde{x}_j t_{B,jk};
\]
therefore, we can solve each column of $\tilde{x}$ in turn by
a back-substitution procedure that involves a triangular
linear solve.  We only run into trouble if one of these systems
is singular (or nearly so), corresponding to the case where $A$
and $-B$ (nearly) have an eigenvalue in common.
