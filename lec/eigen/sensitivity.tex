\section{First-order perturbation theory}

Suppose $A \in \bbC^{n \times n}$ has a simple\footnote{
  An eigenvalue is simple if it is not multiple.
} eigenvalue $\lambda$ with corresponding column
eigenvector $v$ and row eigenvector $w^*$.
We would like to understand how $\lambda$ changes under
small perturbations to $A$.  If we formally differentiate
the eigenvalue equation $A v = v \lambda$, we have
\[
  (\delta A) v + A (\delta v) = (\delta v) \lambda + v (\delta \lambda).
\]
If we multiply this equation by $w^*$, we have
\[
  w^* (\delta A) v + w^* A (\delta v) =
  \lambda w^* (\delta v) + w^* v (\delta \lambda).
\]
Note that $w^* A = \lambda w^*$, so that we have
\[
  w^* (\delta A) v = w^* v (\delta \lambda),
\]
which we rearrange to get
\begin{equation} \label{basic-sensitivity}
  \delta \lambda = \frac{w^* (\delta A) v}{w^* v}.
\end{equation}
This formal derivation of the first-order sensitivity of an
eigenvalue only goes awry if $w^* v = 0$, which we can show is
not possible if $\lambda$ is simple.

We can use formula (\ref{basic-sensitivity}) to get a condition
number for the eigenvalue $\lambda$ as follows:
\[
  \frac{|\delta \lambda|}{|\lambda|}
   = \frac{|w^* (\delta A) v|}{|w^* v| |\lambda|}
    \leq \frac{\|w\|_2 \|v\|_2}{|w^* v|} \frac{\|\delta A\|_2}{|\lambda|}
    = \sec \theta \frac{\|\delta A\|_2}{|\lambda|}.
\]
where $\theta$ is the acute angle between the spaces spanned by $v$ and by $w$.
When this angle is large, very small perturbations can drastically change the
eigenvalue.
