\section*{Deflation}

A sequence of implicit doubly-shifted QR steps with the Francis shift
will usually give us rapid convergence of a trailing 1-by-1 or 2-by-2
submatrix to a block of a Schur factorization.  As this happens,
the trailing row (or two rows) becomes very close to zero.  When the
values in these rows are close enough to zero, we {\em deflate} by
setting them equal to zero.  This corresponds to a small perturbation
to the original problem.

The following code converts a Hessenberg matrix to a block upper triangular
matrix with 1-by-1 and 2-by-2 blocks.  To reduce this matrix further to
real Schur form, we would need to make an additional pass to further
reduce any 2-by-2 block with real eigenvalues into a pair of 1-by-1 blocks.

\lstinputlisting{code/eigen/hessqr.m}

More careful deflation criteria are usually used in practice;
see the book.  This criterion at least corresponds to small normwise
perturbations to the original problem, but it may result in less
accurate estimates of small eigenvalues than we could obtain with
a more aggressive criterion.
