\section{Shifting gears}

The connection from inverse iteration to orthogonal iteration (and
thus to QR iteration) gives us a way to incorporate the shift-invert
strategy into QR iteration: simply run QR on the matrix $A-\sigma I$,
and the $(n,n)$ entry of the iterates (which corresponds to a Rayleigh
quotient with an increasingly-good approximate row eigenvector) should
start to converge to $\lambda - \sigma$, where $\lambda$ is the
eigenvalue nearest $\sigma$.  Put differently, we can run the
iteration:
\begin{align*}
  Q^{(k)} R^{(k)} &= A^{(k-1)} - \sigma I \\
  A^{(k)} &= R^{(k)} Q^{(k)} + \sigma I.
\end{align*}
If we choose a good shift, then the lower right corner entry of
$A^{(k)}$ should converge to the eigenvalue closest to $\sigma$ in
fairly short order, and the rest of the elements in the last row
should converge to zero.

The shift-invert power iteration converges fastest when we choose a
shift that is close to the eigenvalue that we want.  We can do even
better if we choose a shift {\em adaptively}, which was the basis for
running Rayleigh quotient iteration.  The same idea is the basis
for the {\em shifted QR iteration}:
\begin{align}
  Q^{(k)} R^{(k)} &= A^{(k-1)} - \sigma_k I \label{sqr-1} \\
  A^{(k)} &= R^{(k)} Q^{(k)} + \sigma_k I. \label{sqr-2}
\end{align}
This iteration is equivalent to computing
\begin{align*}
  \uQ^{(k)} \uR^{(k)} &= \prod_{j=1}^n (A-\sigma_j I) \\
  A^{(k)} &= (\uQ^{(k)})^* A (\uQ^{(k)}) \\
  \uQ^{(k)} &= Q^{(k)} Q^{(k-1)} \ldots Q^{(1)}.
\end{align*}

What should we use for the shift parameters $\sigma_k$?  A natural
choice is to use $\sigma_k = e_n^* A^{(k-1)} e_n$, which is the same
as $\sigma_k = (\uQ^{(k)} e_n)^* A (\uQ^{(k)} e_n)$, the Rayleigh quotient
based on the last column of $\uQ^{(k)}$.  This simple shifted QR iteration
is equivalent to running Rayleigh iteration starting from an initial vector
of $e_n$, which we noted before is locally quadratically convergent.
