\section{Triangular matrix graphs and groups}

Before moving on from triangular matrices, it is worth pointing
out two useful structural facts.

\subsection{Triangular matrices and DAGs}

Triangular matrices are associated with {\em directed acyclic graphs}
(DAGs), and this is part of what makes them useful for solving linear
systems: forward and back-substitution are both instances of processing
the graph according to a topological ordering by variable dependencies.

In \matlab, a {\em psychologically triangular} matrix is a matrix whose
rows can be permuted to obtain a lower (or upper) triangular matrix.  A
\matlab\ solve involving a psychologically triangular matrix is still
$O(n^2)$, since \matlab\ checks in advance for this structure.  When the
{\tt lu} function in \matlab\ is called without explicitly returning a
permutation, the $L$ matrix it returns is psychologically lower
triangular.

In principle, \matlab\ could use a topological sort to solve a
matrix that could be transformed to triangularity by row {\em and column}
permutations in $O(n^2)$ time.  In practice, it does not!

\subsection{Triangular matrices and groups}

% Psychologically triangular matrices and DAGs
% Matrix groups

The (square) unit lower triangular matrices (lower triangular matrices with
ones on the main diagonal) have several interesting properties:
\begin{itemize}
\item $I$ is unit lower triangular
\item Products of unit lower triangular matrices are unit lower triangular
\item Every unit lower triangular matrix has a unit lower triangular
  inverse
\end{itemize}
That is, the unit lower triangular matrices form an {\em algebraic group}
within the set of square matrices.  If you don't know what an algebraic
group is, that is fine --- but it is worth noting that the unit triangular
matrices are closed under inversion and multiplication.

The invertible lower triangular matrices are also an algebraic group,
as are the unit upper triangular matrices and the invertible upper
triangular matrices.
