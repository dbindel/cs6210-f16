\section{Beyond partial pivoting}

Gaussian elimination with partial pivoting has been the mainstay of
linear system solving for many years.  But the partial pivoting
strategy is far from the end of the story!  GECP, or Gaussian elimination
with {\em complete} pivoting (involving both rows and columns),
is often held up as the next step beyond partial pivoting,
but this is really a strawman --- though complete pivoting fixes
the aesthetically unsatisfactory lack of backward stability in the
partial pivoted variant, the cost of the GECP pivot search is
more expensive than is usually worthwhile in practice.  We instead
briefly describe two other pivoting strategies that are generally
useful: {\em rook pivoting} and {\em tournament pivoting}.  Next week,
we will also briefly mention {\em threshold pivoting}, which is relevant
to sparse Gaussian elimination.

\subsection{Rook pivoting}

In Gaussian elimination with rook pivoting, we choose a pivot at each
step by choosing the largest magnitude element in the first row
{\em or column} of the current Schur complement.  This eliminates
the possibility of exponential pivot growth that occurs in the
partial pivoting strategy, but does not involve the cost of searching
the entire Schur complement for a pivot (as occurs in the GECP case).

For the problem of solving linear systems, it is unclear whether rook
pivoting really holds a practical edge over partial pivoting.  The
complexity is not really worse than partial pivoting, but there is more
overhead (both in runtime and in implementation cost) to handle deferred
pivoting for performance.  Where rook pivoting has a great deal of
potential is in Gaussian elimination on (nearly) {\em singular} matrices.
If $A \in \bbR^{m \times n}$ has a large gap between $\sigma_{k}$
and $\sigma_{k+1}$ for $k < \min(m,n)$, then GERP on $A$ tends to yield
the factorization
\[
  PAQ = LU, U = \begin{bmatrix} U_{11} & U_12 \\ 0 & U_{22} \end{bmatrix}
\]
where $U_{11} \in \bbR^{k \times k}$ and $\|U_{22}\|$ is very small
(on the order of $\sigma_{k+1}$).

Rook pivoting and the closely-related threshold rook pivoting are
particularly useful in constrained optimization problems in which
constraints can become redundant.  Apart from its speed, the rook
pivoting strategy has the advantage over other rank-revealing
factorizations that when $A$ is sparse, as one can often control the
fill (nonzeros in $L$ and $U$ that are not present in $A$).
The LUSOL package of Michael Saunders is a particularly effective example.

\subsection{Tournament pivoting}

In parallel dense linear algebra libraries,
a major disadvantage of partial pivoting is that the pivot search is
a communication bottleneck, even with deferred pivoting.  This is
increasingly an issue, as communication between processors is far
more expensive than arithmetic, and (depending on the matrix layout)
GEPP requires communication each time a pivot is selected.
For this reason, a number of recent {\em communication-avoiding}
LU variants use an alternate pivoting strategy called
{\em tournament pivoting}.

The idea behind {\em tournament pivoting} is to choose $b$ pivot rows in
one go, rather than iterating between choosing a pivot row and
performing elimination.  The algorithm involves each processor proposing
several candidate pivot rows for a heirarchical tournament.  There are
different methods for managing this tournament, with different levels of
complexity.  One intriguing variant, for example, is the remarkable
(though awkwardly named) {\tt CALU\_PRRP} algorithm, which uses
rank-revealing QR factorizations to choose the pivots in the tournament.
The {\tt CALU\_PRRP} algorithm does a modest amount of work beyond what
is done by partial pivoting, but has better behavior both in terms of
communication complexity and in terms of numerical stability.
