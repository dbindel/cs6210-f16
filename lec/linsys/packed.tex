\section{Packed storage}

In $LU$ factorization, the locations where we write the multipliers in
$L$ are exactly the same locations where we introduce zeros in $A$ as we
transform to $U$.  Thus, we can re-use the storage space for $A$ to
store both $L$ (except for the diagonal ones, which are implicit) and
$U$.  Using this strategy, we have the following code:
\begin{lstlisting}
%
% Overwrite A with L and U factors
%
function [A] = mylu(A)
  n = length(A);
  for j=1:n-1
    A(j+1:n,j) = A(j+1:n,j)/A(j,j);
    A(j+1:n,j+1:n) = A(j+1:n,j+1:n) - A(j+1:n,j)*A(j,j+1:n);
  end
\end{lstlisting}
If we wanted to extract the $L$ and $U$ factors explicitly, we could
then do
\begin{lstlisting}
  LU = mylu(A);
  L = eye(length(A)) + tril(A,-1);
  U = triu(A);
\end{lstlisting}

The bulk of the work at step $j$ of the elimination algorithm is in
the computation of a rank-one update to the trailing submatrix.
How much work is there in total?  In eliminating column $j$, we do
$(n-j)^2$ multiplies and the same number of subtractions; so in all,
the number of multiplies (and adds) is
\[
  \sum_{j=1}^{n-1} (n-j)^2 = \sum_{k=1}^{n-1} k^2 = \frac{1}{6} n^3 + O(n^2)
\]
We also perform $O(n^2)$ divisions.  Thus, Gaussian elimination, like
matrix multiplication, is an $O(n^3)$ algorithm operating on $O(n^2)$ data.
